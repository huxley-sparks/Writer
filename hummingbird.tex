\title{Hummingbird}
\author{Huxley Sparks}
\begin{docuemnt}
	
		It's such a rarity to see a hummingbird levitate near, with its wings moving rapidly. For what's worth, the bird is a paradox. Always in a rush but 
	curious as a child. The bird looks at the near object, approaches, retreats and observes. From there, its choice is obvious, to leave or approach. Being in the
	forest and the object being a moon flower, she grabs the nectar using her beak. A moment later, she transcends over the forest and escapes sight.

		On the trail surrounded by vegetation of a swamp on a pit in California, I descend further into its core. Squirrels make most of the noise as I 
	approach and in the background, the constant annoying presence of the freeway. I turn a deaf ear to the cars as I continue on the trail, somewhat wondering.
	The less I think of cars, the easir I can relax. Every time I think of the metal box, I remember all the things I have to do, the things I'm obligated to do.
	Even when away, the nagging stays clingged to me.

		Moving further, I see that the trail is intercepted by a deep stream. I continue to walk foward until my feet reach the shore. The path continues on
	the other side; at least 4 meters away. Unwilling to return early, I look around for an alternative. On my left, a dead tree with a trail on the other side.
	Off the standard direction; I take it to keep going. Over the carcass and along the stream which becomes shallow and populated by rocks that break its surface.
	
		Boots on smooth rocks of the stream make it towards the other side. A crouching tree greets me, extending its branches to give me a hug. Ducking
	under its arms, I pass him and see the route near. I run up the waist high incline and return to the path.

		In my pants lies my phone and I get an urge of checking it. Really, do I hae to check on it after half an hour? Leaving my eyes opened ajar in response
	to my dependency, I take a deep breath in, lower my shoulders and try to walk upright. Looking to the forest to distract me, signs make it worse. Signs, 
	cement, tagging on trees. Makes me think that we're supposed to see this as a theme park ride, or the bathroom. Leaving the reminders behind me; I walk
	straight ahead.	

		Another stream, cars sound faint and about 10 squirrels rustle in the grass. The ground rises to an incline, 45 degrees mostly, and I begin to wonder
	about the flies attraction to human heads. I doubt it gives them more nutrients than a plant. Knocking a few away from me, they seem to leave me alone.
	Almost at the zenith, I notice how I can't hear the constant nagging anymore; how I'm completely surrounded by the forest. My shoulders drop lower.

		It doesn't take long for someone to intrude, especially with a leafblower. The noise is faint but its presence appearant. If it lies ahead, there is
	no reason to keep going. After analysis, I figure out that the noise is outside of the forest, a person doing lawn work near it. Despite its surroundings, I
	go further into it.

		A few more squirrels scatter, some leaving a meter before and others when my feet pass them in centimeters. My head starts to feel heavy, I should go
	back, what if its serious? While I'm back there, I should watch over Jack, go to my job, fix my roommates computer and pay my bills. No! No, that can wait,
	it will always be there until I end. My back foot tries to pivot backwards, twisting only half way before being yanked foward.

		Rocks less in area than my boots, surrounded by stream, almost on the other side. With one foot in the air, the anchored foot behind me loses its grip
	on the rock and goes in the air. My body stays up as the other foot goes to premature landing, in front of me, slanting slightly.

		Minutes later, the path expands to the length of a car. The wind moving leaves just gives the leafblower more emphasis. Without it, the forest is
	silent, no squirrels scattering, nothing flying to to my face, nada. What am I getting for walking here and then it pops into my head. The path isn't a road
	with yellow bricks and no prize at the end, so why do it?

		Voices of his dad become a vex for him; screaming at him for neglible details, talking of him behind walls; disapproving his methods of teaching,
	always telling he's doing nothing. Containing the pressure in his head by his hands, he takes a deep breathe in, exhales by extending his hands to his waist
	level. Opening his eyes and seeing the forest, lets his worries scatter.

		I couldn't tolerate my dad anymore, the anger of my dad always filtered on the family, becoming more money dependent, misery more apparent on his face,
	the increasing frequency of drinking and the more commiment he has for religion. Going to Church on Sunday, whenever he durnk too much on Saturday, always
	making the whole family go as we were the ones who made him drink. The only thing that keeps him sane is the house and its maintenance. When the dad isn't
	working on the house, he is sitting in fornt of the television, watching the images flash every second.

		The boy knows he owes his dad a lot but not complete obedience. His dad was a once hard working bull, leaving everything in his work and being relaxed
	while at home. The dad now does anything to keep him busy and away from thinking. From the distance, he can hear his dad complain about something not being
	done. He ignored him when he went to his empty ramblings.
	
		"I'm stuck with them, all of them." To desert them without considering what they did for him would be foolish. Looking up to the trees for an answer,
	he sees a bird land on a nest. Three little birds pop up, chirping for food. The mom leaves them a worm and then departs. He foolows the bird as it goes on
	a stone, putting its head in the water, shaking the wetness. He smiles and thinks about visiting them on Thanksgiving and forth of July.

		Walking with less weight than before, he notices the variety of the path disappear, leaving only a straight path with a few trees reaching over it.
	Fatigue does set in, his eyes fall down to the ground and it presents itself. This isn't the life he wants, born to a middle class family, being surrounded
	by a population that plays simon-says with the television, finding nothing worth dying for. Life leaves him with a bland taste in his mouth and redoing it 
	would  be the same. He hasn't created anything, destroyed flawed theories, only to live within his means.

		Not standard to come to the words but to stay on the path made. What else is ahead of him, a job for 40 years, old age and being nice? The default
	sounds replusive, enough to bring his hands to his knees and cough. To get married so he doesn't feel alone while growing older, to have children to have
	a sense of purpose and legacy. Once he had pushed everything away, he was completely alone. No machines made noise, the wind being the only presence near
	him. He tried to think of what he actually wanted; to be rich? For what?

		Empty for a while, he tried to derive his wants from his hatred. Why did he hate the standard? There were so many things in the package to keep him
	occupied, even if he did get bored easily. If he didn't like his work, he can focus on his family. If not them, a hobby. If unwilling, he can keep working 
	until he retires and live off his security. It was too undecisive. His first virtue discovered was simplicity. To apply it, he had to be willing to dedicate
	most of his life to one passion. To lose curiousity with the world except for his scope. Even if his efforts are in vain, his character remains consistent.

		Now being able to lift his head, he sees a red brick wall in his way. The dead end of the path. His thoughts come to a stop and he falls back to
	reality. He holds his hand against the wall and feels the giant's push as he pushes against it. This is as far as he's allowed to go.

		He continues to stand there as the sun starts to set, he and the wall by and by. A tree looks down on them as it blocks their view. Unwilling to stay
	in his spot, he takes a step.
