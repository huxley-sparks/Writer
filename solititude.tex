\title{Solitude}
\author{Huxley Sparks}
\begin{document}
		Hands tucked in his jacket, head down and his body being held by a thick garment; Marlow walks by the pond. Being night, he is the only one who roams
	under the orange street lights that are indifferent to his company. With the same emotion, Marlow hardly notices the lights except for those that guide his
	way. Trees over his head, trying to reach the pond next to him; a full moon makes the night less menacing, animals sleeping or leaving their homes in search
	of food. The sound that persists is the pond, its gentle current creating a soothing sound.

		The pond is a stranger to Marlow and this was the first time he had no intention of returning home. Everything is different for him now, he knew
 	he had to look for a specific city anchors in order to return to his room. To go back home for any whim was not an option, walking over there, impossible.
	He was more immune to his whims than he thought, it never interrupting his work or study. What did cause him to pause were memories, a yearning for the past.

		At dinner, he sat at a table with others, the older folk talking about their work and how much they hate it or when they will go over the weekend. A
	girl similar to his age is always connected to her phone, with either her or it doing the pulling. Marlow had grew accustomed to his dinner's ambience so when
	it became quiet lately, it felt somehow empty. Again, not as bad to the point of crying but he did miss it. Then he thought of them, he unable to interrupt
	their flow from now on. There were times they tried to include him but he could never have anything that would appeal to them. His strategy when the atttention
	shifted to him is to cause a stir in others so he or she will ask for the light. Then Marlow would fade to the shadows and finish his meal while being aware
	of their talk.	

		A gray fog crawling on the ground engulfs his feet and Marlow puts his head upright to see if it can consume his body. If so, change course. Benign,
	a small fog overflowing from the pond, it would only be threat if clouds from the sky fell. Walking along, Marlow hits a few trees that hang low on the trail
	as he looks at the stew brewing from the pond. It reminds him of his mom cooking beans and it foaming after boiling. From her making standard beans to black
	beans, red kidney beans, garbanzo and pinto beans. He always prefered the black beans over the pinto, having ate too much of the pinto at home. What he
	couldn't believe is how much better the beans tasted after he started to make them.

		Learning to keep his eyes on the path while still taking a few glances at the lake, Marlow walks with his hands out and with a slower gait. Continuing
	on his path, he notices the trees being more omnipresent, covering the sky and letting the shadows become darker. Marlow stops to look at what lies ahead, a
	feeling of insecurity fills him. "There is no one to come to my rescue, these are unmarked territory according to yourself, just like this place was." Marlow
	then resumes his walking without saying good-bye to the lights.

		More silent than he expected walking in the dark corridor of trees. After a while, he started to think again, this time about why he was there. He 
	thought that there would be a pull or a calling or something but no. Actually, to come over here, he had to go out of his way to get the loan and do some
	planning. It was only until he was at the airport that he realized how his life changed and to go back would be meek of him. He stayed en route since he
	always preferred his choice.

		The naked branches exert themselves to the sky with awkward separation, fingers bending in ways beyond natural. Many arms and long fingers hide most
	of the gray sky from him so he can be fully emersed in the dungeon. A few orange lights in shapes of tourches only show the path, too weak to challenge the
	darkness lingering in the woods. Passing one light, his shadow becomes a gaint and in these shadows, they seem to manifest in black. One of his feet refuse
	to move foward, anchoring itself to the cement. He pays more attention to the shadow that seems to become more real. Unwilling to confront the challenge 
	while being stuck, he gives a hard tug to the anchored foot and lands it, closer to the giant.

		The shadow bends down to him and whispers to him with the strength of the wind. Fighting through the uncomprenhendable language, Marlow makes it all
	the way towards the feet of the giant. Barely the geight of the trees, Marlow's expectation of it fails. Without reason, Marlow sees the giant swing at him.
	He jumps to his side and watches the giant hit the wind. Lifting his self up with his palms on the floor, noticing the rocks near him. Checking where the giant
	is, just two meters away from him, he picks up some rocks and gets further way.

		Both of them facing the other, Marlow starts to think that this is a stupid idea. "Rocks, no. It won't do much, it won't even feel it. What makes you
	so sure --" Marlow jumps foward, barely escaping the punch sent straight tot he floor. Quickly getting up, he gets further away. "What the hell? I don't
	have anything to fight a giant. If I just had a gun, I could probably do something, at least signal for help. But no, all I have are rocks."

		Moving in parallel to the path, Marlow moving foward by walking backwards and the giant following at the same pace, a flag goes up in his head. Right
	away he can't see what is inconsistent. "What am I trying to disprove?" He closes his eys to think and then realizes it, the giant isn't making any noise.
	Its not even causing the earth to shake or leaving prints on the ground. He opens his eyes and looks at where the punch landed, the area unsacthed. "Maybe
	it has force but I know for certain now that its light." Squeezing the rocks in his hands, Marlow cocks his arm back in a pitcher's stance and throws the
	rocks at the giant. Several black birds fly away.

		A few seconds later, he hears the rocks hit the ground and the giant is gone. Exhaling, Marlow walks back to the rocks to get a better view of what
	happened. Looking around, he sees several birds, not crows, on the branches, sitting still. All of them have their backs facing him, none of them acknowledging
	his presence. Picking up a rock, he thinks about rying to hit one of the birds. What goes through his head is what if he hits one of them, misses; their
	response? Unwilling to provoke the evil in the birds, he drops the rocks and continues his route.

		After a turn, light becomes more visible and a hill lies along his path. He maintains his pace, letting his eyes wander to the current setting. The 
	arms try to hold on to the sky but for most, it escapes them. For those that couldn't hold on, leaves fill them. The trees now become separate entities, each
	displaying their distinct colors. An ambience of life fills his ears as he approaches the end of the woods and the beginning of the ascend. The last steps out
	keep their pattern as the mind changes, trying to analyze what just happened. Beides causing a conflict in his head, he continues walking, no point of 
	stopping in the dark. As he goes up the hill, he keeps looking foward, to the top of the hill and imagines himself there already.

		Walking up the path, he starts to notice how deep the incline is. To run this path would get you as far as walking so no point of making haste. Each
	step makes little progress, being close from the other. His whole body yearns to go back, its so much easier to go back, so natural. Exherting his head foward,
	his body resentfully follows.

		A while later, he finds himself being surrounded by clouds. This causes him to go back slightly. He closes his eys and keeps walking; "I have to reach
	the top." more goes against him as the air gets thinner. Warnings keep flashing in his head, reminding him his body is fragile and requires rest to function.
	"And of course the body's natural position is to lay down and its only reason to move is to get food." His lungs squeeze harder to function, lethargy spreads
	quickly through his body as he fights to make the last steps towards the top. Danger rining sirens, flashing red, his perception becomes blurred, taking his
	last breath, he falls over the step.

		An indeterminate amount of time passes when he regains consciousness. Eyes still shut, he turns himself over so he face the sky. Breathing, he feels
	a splash of blue bliss press on his lips. Opening his eyes, he finds no one, nothing near him. Standing up, he realizes his hindrance is gone, being at the top
	with the conditions of being at sea level. With his area empty, Marlow looks back to the woods he went through. "Its hell, all evil lurks there and you were
	luckly to get out alive." He lowers his eyelids and looks to his left to his imaginary self, "being evil, it will never possess the strength of good. That
	place deserves to burn." Marlow sighs as he returns to the woods and then looks to the park besides it. "Things are not naturally evil, evil is what we deem
	to be evil and the way we treat it, it fulfills our destiny for it." then he recalls why the forest was called evil, the death of a child.

		The story dates back at least a century ago, when a group of friends found the body. It became the most important news for the city. From the marks
	on him, people assumed he was killed by the animals from the park. Everyone pitied the parents, giving them money to console the terrible vex on their heads.
	Then as usual, it happened, everyone moved on. There were pieces that were still there but the woods were assigned the role of executioner. Less people
	started using the route, less reason to keep it clean, more reason to see the darkness around it.
		
		A few year later, cops arrested the parents for child abuse. The police decide to keep this away from the media as they quietly transfer the parents
	to an out of state prison.
\end{document}
